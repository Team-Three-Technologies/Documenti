\documentclass[a4paper,12pt]{article}
\usepackage{docsTemplate}
\usepackage{longtable}

\setTitle{Verbale Interno 23/10/2025}
\setAuthors{Sara Gioia Fichera}
\setVerificators{Filippo Compagno}
\setApprovation{Bianca Zaghetto}
\setVersion{1.0}
\setType{Interno}
\setPlace{Discord}
\setTime{16:00}
\setDestination{Team Three Technologies}
\setAbsents{Mattia Oliva Medin}

\begin{document}
\include{memoTitlepage}

\newpage
\section*{Registro delle modifiche}

\begin{table}[h!]
\rowcolors{2}{lightgray}{white}
\begin{tabularx}{\textwidth}{|l|l|l|l|L|L|}
\hline
\textbf{Vers.} & \textbf{Data} & \textbf{Autore} & \textbf{Ruolo} & \textbf{Verifica} & \textbf{Oggetto} \\
\hline
\textbf{1.0} & 24/10/25 & Bianca Zaghetto & Responsabile &  & Approvazione documento \\
\textbf{0.1} & 23/10/25 & Sara Gioia Fichera & Amministratore & Filippo Compagno & Stesura completa \\
\hline
\end{tabularx}
\end{table}

\newpage
\tableofcontents
\newpage

\markright{}
\newpage
\section{Ordine del giorno}

\begin{itemize}
    \item Discussione a seguito del colloquio con Miriade
    \item Preparazione al Diario di Bordo
\end{itemize}

\section{Svolgimento}
\rowcolors{2}{white}{white}
\begin{longtable}{P{5cm}|P{10cm}}
\textbf{Colloquio con Miriade} & L’incontro con Miriade è stato considerato utile per comprendere meglio le caratteristiche del capitolato, tuttavia, il gruppo ha rilevato una minore affinità con la proposta rispetto ad altre opportunità valutate.\\
\\
\textbf{Diario di Bordo} & Il gruppo ha delineato i progressi e le difficoltà rilevate in previsione del Diario di Bordo del 27 ottobre 2025, che sarà presentato in aula da Francesco Balestro.\\
\\
\textbf{Creazione Github Pages} & Filippo Compagno ha presentato un primo aggiornamento riguardante la creazione di una GitHub Pages finalizzata a rendere accessibile online la pagina di presentazione del gruppo. Ha mostrato le sezioni già impostate e le funzionalità implementate fino a questo momento; il gruppo ha espresso una valutazione positiva del lavoro svolto e non ha evidenziato criticità. \\
\\
\textbf{Discussione sulla modalità di versionamento} & È stata discussa la modalità di versionamento da adottare per la documentazione di progetto, da adottare nel registro delle modifiche. Il gruppo ha concordato di utilizzare una notazione del tipo \textit{approvazione.modifica} (ad esempio 1.0, 1.1), perchè ritenuta più chiara e immediata per identificare le diverse versioni dei file. Tale convenzione consente di distinguere in modo esplicito le revisioni approvate dalle semplici modifiche, garantendo uniformità e tracciabilità all’interno della documentazione del progetto.
\\
\\
\textbf{Organizzazione e divisione compiti} & Sara Gioia Fichera è incaricata della redazione del presente verbale, mentre Francesco Balestro, Filippo Compagno e Nenad Radulovic ne curano la verifica.
Nenad Radulovic è incaricato della redazione della prima stesura del documento relativo al Way of Working.
Francesco Balestro provvederà alla creazione delle slide per il prossimo Diario di Bordo, utilizzando il template già predisposto.
Infine, Sara Gioia Fichera sarà responsabile della redazione del verbale esterno relativo all’incontro con l’azienda Miriade.
\\
\end{longtable}

\section{Decisioni}

\rowcolors{2}{lightgray}{white}
\begin{tabularx}{\textwidth}{|l|X|}
\hline
\textbf{Assegnatario} & \textbf{Task Todo} \\
\hline
\textbf{Francesco Balestro} & Creazione Power point per Diario di Bordo \\
\hline
\textbf{Sara Gioia Fichera} & Stesura verbale esterno con Miriade
\\
\hline
\textbf{Andrea Masiero} &Stesura del documento di analisi dei capitolati\\
\hline
\textbf{Nenad Radulovic} & Stesura del documento di Way of working\\
\hline
\end{tabularx}

\end{document}
