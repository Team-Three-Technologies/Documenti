\documentclass[a4paper,12pt]{article}
\usepackage{docsTemplate}
\usepackage{eurosym}

\setTitle{Dichiarazione di Impegni}
\setAuthors{Filippo Compagno}
\setVerificators{Bianca Zaghetto}
\setApprovation{Sara Gioia Fichera}
\setVersion{1.0}
\setType{Esterno}
\setDestination{Tullio Vardanega \\ & Riccardo Cardin \\ & Team Three Technologies}

\begin{document}
    \begin{titlepage}

    \begin{minipage}{0.4\textwidth}
        \includegraphics[width=\linewidth]{logo.png}
    \end{minipage}
    \begin{minipage}{0.7\textwidth}
        \raggedright
        \Large\textbf{Team Three Technologies}\\
        \large{t3.unipd@gmail.com}\\
    \end{minipage}
    \vspace{1,5cm}    
    \begin{center}
        {\Huge \textbf{\docTitle}\par}
        \vspace{0.6cm}
        {\Large Progetto di Ingegneria del Software\par}
        \vspace{0.3cm}
    \end{center}
    
    \vfill
    
    \noindent
    \begin{tabular}{r|l}
        \textbf{Redazione} & \docAuthors \\
        \textbf{Verifica} & \docVerificators \\
        \textbf{Approvazione} & \docApprovation \\
        \textbf{Versione} & \docVersion \\
        \textbf{Tipo} & \docType \\
        \textbf{Destinatari} & \docDestination \\
    \end{tabular}
    
    \vfill
    
    \noindent\rule{0.7\textwidth}{0.4pt}\\[0.5cm]
    \large\textbf{Università degli Studi di Padova}\\
    \large Corso di Laurea in Informatica\\
    \small A. A. 2025/26

\end{titlepage}

    \section*{Registro delle modifiche} {
        \begin{table}[h!]
            \rowcolors{2}{lightgray}{white}
                \begin{tabularx}{\textwidth}{|l|l|l|l|L|L|}
                \hline
                \textbf{Vers.} & \textbf{Data} & \textbf{Autore} & \textbf{Ruolo} & \textbf{Verifica} & \textbf{Oggetto} \\
                \hline
                \textbf{1.0} & 30/10/25 & Sara Gioia Fichera & Amministratore &  & Approvazione documento \\
                \hline
                \textbf{0.2} & 29/10/25 & Filippo Compagno & Analista & Bianca Zaghetto & Completamento dei contenuti del documento \\
                \hline
                \textbf{0.1} & 24/10/25 & Filippo Compagno & Analista & Bianca Zaghetto & Creazione della struttura e prima stesura del documento \\
                \hline
            \end{tabularx}
        \end{table}
    }
    \newpage
    
    \tableofcontents
    \newpage
    
    \section{Introduzione} {
        Lo scopo di questo documento è quello di esporre l'impegno orario che ogni membro del terzo gruppo, \textit{Team Three Technologies}, ha ritenuto di poter garantire ai fini della realizzazione del progetto relativo al \textbf{capitolato C3 - "DIPReader"}, proposto da \textit{Sanmarco Informatica S.p.A.}, per cui il gruppo ha deciso di candidarsi.\\
        Vengono inoltre esposti il preventivo dei costi e la data di consegna prevista per il completamento del progetto.
    }

    \section{Ruoli nel progetto} {
        I ruoli che ogni membro del gruppo assumerà durante il corso del progetto sono i seguenti:
        \begin{itemize}
            \item \textbf{Responsabile}\\
                Si occupa della gestione del gruppo, del coordinamento delle attività e della comunicazione con il proponente e il committente.\\
                È una figura essenziale durante tutto l'arco del progetto, quindi con una presenza stabile che però va a decrescere con la maggiore autonomia che i membri assumono. 
            \item \textbf{Amministratore}\\
                Si occupa della gestione dell'infrastruttura IT utilizzata dal gruppo.\\
                La sua presenza sarà costante per tutta la durata del progetto, per garantire il corretto funzionamento degli strumenti utilizzati e per fornire supporto tecnico ai membri del gruppo.
            \item \textbf{Analista}\\
                Si occupa della raccolta e dell'analisi dei requisiti del progetto.\\
                Una volta completata questa fase è previsto che il suo impegno si riduca notevolmente, intervenendo solo in caso di modifiche ai requisiti o di necessità di chiarimenti.
            \item \textbf{Progettista}\\
                Colui che si occupa della progettazione dell'architettura del software e della definizione delle specifiche tecniche.\\
                Il suo lavoro si basa su quello svolto dall'analista e guida quello del programmatore, quindi gli è garantita una presenza alta per poter lavorare a stretto contatto con entrambi i ruoli e per garantire un prodotto con un design che rispetti i requisiti.
            \item \textbf{Programmatore}\\
                Colui che si occupa della scrittura del codice sorgente del software.\\
                La sua presenza sarà concentrata nella fase di sviluppo del \textit{Proof of Concept} (\textit{PoC}) e del \textit{Minimun Viable Product} (\textit{MVP}), quindi a seguito di una buona parte del lavoro svolto durante le fasi di analisi e di design.
            \newpage
            \item \textbf{Verificatore}\\
                Colui che si occupa della verifica di tutto ciò che viene prodotto, dalla documentazione al software, per garantire che questi rispettino i requisiti stabiliti.\\
                La sua presenza sarà costante per tutta la durata del progetto.
        \end{itemize}
        Abbiamo previsto che i ruoli ruotino tra i membri in corrispondenza dell'inizio di ogni sprint, quindi ogni 2 settimane senza contare possibili ritardi.
    }

    \section{Valutazione dell'impegno e dei costi} {
        \subsection{Impegno dei membri} {
            Ogni membro del gruppo si impegna a dedicare un totale di \textbf{92 ore produttive}, che saranno suddivise tra i vari ruoli secondo come segue:\\
            \begin{table}[h!]
                \centering
                \begin{tabularx}{\textwidth}{|X|c|c|c|c|c|c|c|}
                    \hline
                    \rowcolor{lightgray}
                    \multicolumn{1}{|c|}{} & \textbf{RE} & \textbf{AM} & \textbf{AN} & \textbf{PG} & \textbf{PR} & \textbf{VE} & \textbf{Totale} \\
                    \hline
                    Francesco Balestro & 9 & 9 & 12 & 17 & 24 & 21 & 92 \\
                    \hline
                    Filippo Compagno & 10 & 9 & 11 & 17 & 23 & 22 & 92 \\
                    \hline
                    Sara Gioia Fichera & 9 & 9 & 11 & 19 & 23 & 21 & 92 \\
                    \hline
                    Andrea Masiero & 9 & 9 & 11 & 19 & 23 & 21 & 92 \\
                    \hline
                    Mattia Oliva Medin & 9 & 9 & 12 & 17 & 24 & 21 & 92 \\
                    \hline
                    Nenad Radulovic & 9 & 10 & 11 & 17 & 24 & 21 & 92 \\
                    \hline
                    Bianca Zaghetto & 9 & 9 & 12 & 18 & 23 & 21 & 92 \\
                    \hline
                    \textbf{Totale} & \textbf{64} & \textbf{64} & \textbf{80} & \textbf{124} & \textbf{164} & \textbf{148} & \textbf{644} \\
                    \hline
                \end{tabularx}
                \caption{Ore di ogni componente per ruolo}
            \end{table}
            \\
            Nella tabella vengono utilizzate le seguenti abbreviazioni: RE per Responsabile, AM per Amministratore, AN per Analista, PG per Progettista, PR per Programmatore, VE per Verificatore.\\
        }
        \newpage
        \subsection{Costo dei ruoli} {
            I costi di ciascun ruolo sono riassunti nella seguente tabella:
            \begin{table}[h!]
                \centering
                \begin{tabularx}{\textwidth}{|X|c|c|c|}
                    \hline
                    \rowcolor{lightgray}
                    \textbf{Ruolo} & \textbf{Costo Orario} & \textbf{Ore} & \textbf{Costo Totale} \\
                    \hline
                    Responsabile & 30 \euro/h & 64 & 1920 \euro \\
                    \hline
                    Amministratore & 20 \euro/h & 64 & 1280 \euro \\
                    \hline
                    Analista & 25 \euro/h & 80 & 2000 \euro \\
                    \hline
                    Progettista & 25 \euro/h & 124 & 3100 \euro \\
                    \hline
                    Programmatore & 15 \euro/h & 164 & 2460 \euro \\
                    \hline
                    Verificatore & 15 \euro/h & 148 & 2220 \euro \\
                    \hline
                    \textbf{Totale} &  & \textbf{644} & \textbf{12980 \euro} \\
                    \hline
                \end{tabularx}
                \caption{Costo per ruolo}
            \end{table}
        }
    }

    \section{Preventivo} {
        In seguito alla suddivisione delle ore produttive tra i vari ruoli e dopo aver considerato l'impegno dei vari membri del gruppo si preventiva un costo finale di progetto pari a \textbf{12980,00 \euro}, come verificabile dalla sezione precedente.

    }
    
    \section{Consegna} {
        Il gruppo prevede di consegnare il prodotto finale relativo al capitolato C3 - \textit{"DIPReader"} entro e non oltre la data del \textbf{27 Marzo 2026}.\\
        Il tempo stimato è quindi di circa 21 settimane e comprende un margine che serve a compensare evenutali ritardi dovuti a festività e impegni universitari o possibili imprevisti.\\
        Una prima ipotesi è quella di dedicare indicativamente circa 11 settimane allo sviluppo del \textit{PoC} e le restanti 10 a quello del \textit{MVP}.
    }

\end{document}